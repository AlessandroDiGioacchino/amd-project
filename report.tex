
\documentclass{article}


\usepackage[T1]{fontenc}
\usepackage[english]{babel}
\usepackage[margin=2cm]{geometry}
\usepackage[utf8]{inputenc}

\usepackage{amsmath, amssymb}
\usepackage{caption}
\usepackage{comment}
\usepackage{derivative}
\usepackage{float}
\usepackage{minted}
\usepackage{xcolor}

\usepackage[colorlinks]{hyperref}  % ALWAYS load this package LAST


\definecolor{LightGray}{gray}{0.9}


\title{“Finding similar items” project \\ \small Algorithms for Massive Datasets}
\author{Alessandro Di Gioacchino}
\date{January 2024}


\begin{document}

  \maketitle

  \tableofcontents

  % \begin{comment}
    % The project report, preferably written in LaTeX, will be evaluated
    % according to the following criteria:
    % - correctness of the general methodological approach,
    % - replicability of the experiments,
    % - correctness of the approach,
    % - scalability of the proposed solution,
    % - clarity of exposition.
    %
    % The report should contain the following information:
    % - the chosen dataset, and the parts of the latter which have been
    %   considered,
    % - how data have been organized,
    % - the applied pre-processing techniques,
    % - the considered algorithms and their implementations,
    % - how the proposed solution scales up with data size,
    % - a description of the experiments,
    % - comments and discussion on the experimental results.
    %
    %
    % The report must also contain the following declaration: “I declare
    % that this material, which I now submit for assessment, is entirely my
    % own work and has not been taken from the work of others, save and to the
    % extent that such work has been cited and acknowledged within the text of
    % my work.
    % I understand that plagiarism, collusion, and copying are grave and
    % serious offenses in the university and accept the penalties that would be
    % imposed should I engage in plagiarism, collusion or copying. This
    % assignment, or any part of it, has not been previously submitted by me or
    % any other person for assessment on this or any other course of study.”
  % \end{comment}

  \vspace*{\fill}
  \textit{I declare that this material, which I now submit for assessment, is
  entirely my own work and has not been taken from the work of others, save and
  to the extent that such work has been cited and acknowledged within the text
  of my work.} \\
  \textit{I understand that plagiarism, collusion, and copying are grave and
  serious offenses in the university and accept the penalties that would be
  imposed should I engage in plagiarism, collusion or copying. This assignment,
  or any part of it, has not been previously submitted by me or any other
  person for assessment on this or any other course of study.}

  \newpage

  \section{Dataset}
  The dataset is a subset of Yelp's businesses, reviews, and user data; the
  considered version is.. \\
  It contains five JSON files (for simplicity, the
  \texttt{yelp\_academic\_dataset\_} common prefix is omitted):
  \texttt{business}, \texttt{checkin}, \texttt{review}, \texttt{tip}, and
  \texttt{user}; in particular, the one used in this project is
  \texttt{review}, with a size of roughly five gigabytes. Only the first
  \( 500000 \) lines of the file have been considered, and among these a
  smaller random sample of \( 1000 \) elements has been extracted: this
  workaround is required when working with the free version of Google Colab.
  Although the entire dataset could fit within the provided \( 13 \) GB of RAM,
  some operations of the Spark MapReduce framework would most likely timeout
  (currently, a free Colab notebook can run for at most \( 12 \) hours). \\
  The dataset has nine columns:
  \begin{itemize}
    \item \texttt{review\_id} is a unique string characterizing each review;
    \item \texttt{user\_id} is a string characterizing the user who has written
    the review – the same user can write different reviews;
    \item \texttt{business\_id} is a string characterizing the business whose
    review has been written – the same business can be reviewed multiple times;
    \item \texttt{stars} is the overall rating the user has given to the
    business in the specific review, ranging from one to five;
    \item \texttt{useful}, \texttt{funny}, \texttt{cool} are the amounts of
    respective reactions the review has received from users;
    \item \texttt{text} is a string containing the review itself;
    \item \texttt{date} says when the review has been written (or published).
  \end{itemize}
  Once again, only a subset of columns is relevant for the goal of finding
  similar reviews: \texttt{review\_id} and \texttt{text}. Actually, only the
  \texttt{text} column is necessary, as the items we are checking for
  similarity of is the very text of these reviews; any review identifier could
  be used to store the mapping between text and review, and indeed a simple
  counter could be used in place of the alphanumeric \texttt{review\_id}. One
  can argue the review text is enough to identify a review: unfortunately, it
  requires further processing before being fed to the algorithm, and storing a
  mapping between the raw and the processed version would be expensive.
  Moreover, we do not have any guarantees on the uniqueness of review texts,
  and in fact detecting duplicated reviews, which could be exploited to boost
  the overall opinion of a business, is one of the purposes of finding similar
  items.

  \section{Pre-processing}
  Before we consider the actual pre-processing, it should be mentioned that
  both the flavors of sampling (with and without replacement) can be useful:
  the former allows to explore a larger space, the latter is going to tell us
  whether the algorithm implementation is correct. Indeed, if the result also
  contains some exact duplicates, our confidence in the correctness of the
  implementation increases. To this end, it is important to note we should not
  use a standard sampling with replacement, because the probability of an
  element to be drawn twice is very low (\( \slashfrac{ 1 }{ 500 } \)); we can
  make sure some elements of the sample are duplicated with the following
  approach.
  \begin{enumerate}
    \item Extract the initial sample of size \( s \);
    \item fix some number \( a \leq \lfloor \slashfrac s 2 \rfloor \) of
    elements to replace;
    \item generate \( a \) pseudo-random numbers \( d_1, \, \dots, \, d_a \) in
    \( [ 1, \, s ) \);
    \item remove elements in positions \( d_1, \, \dots, \, d_a \) from the
    sample;
    \item duplicate elements in positions \( d_1 - 1, \, \dots, \, d_a - 1 \),
    occupying the indexes that were freed in the previous step.
  \end{enumerate}
  This way we get a sample of size \( s \) with \( a \) duplicates. \\
  The pre-processing techniques applied to the dataset are pretty standard when
  working with text elements. \\
  The first step is removing stop words, i.e. words that are filtered out
  because they are not significant to our analysis. Keeping in mind that there
  is no one-fits-all list of stop words, what was used in the project is the
  list provided by package “Natural Language Toolkit” (\texttt{nltk}). Some
  stop words contained in this list are \texttt i, \texttt{me}, \texttt{my},
  \texttt{myself}, \dots \\
  %
  \begin{minted}[
    frame=single,
    framesep=2mm,
    baselinestretch=1.2,
    bgcolor=LightGray,
    fontsize=\footnotesize,
    linenos
  ]{python}
    import nltk
    from nltk.corpus import stopwords

    nltk.download('stopwords')
    stopwords.words('english')[:12]
  \end{minted}
  %
  \begin{minted}[
    frame=leftline,
    framesep=2mm,
    baselinestretch=1.2,
    bgcolor=LightGray,
    fontsize=\footnotesize,
    ]{python}
    ['i',
    'me',
    'my',
    'myself',
    'we',
    'our',
    'ours',
    'ourselves',
    'you',
    "you're",
    "you've",
    "you'll"]
  \end{minted}
  %
  As we can see, it only contains lowercase letters, so we first need to apply
  a transformation to the dataset, turning all letters in their lowercase
  counterpart. Also, elements of the dataset are entire reviews, so we convert
  them to list of words by splitting them on any white space character; this
  also comes with the nice side effect of removing consecutive white space
  characters. Finally we filter away stop words. \\
  %
  \begin{minted}[
    frame=single,
    framesep=2mm,
    baselinestretch=1.2,
    bgcolor=LightGray,
    fontsize=\footnotesize,
    linenos
  ]{python}
    # Turn all letters to lowercase
    id_txt_lower = id_txt_rdd.map(
      lambda id_txt: (id_txt[0], id_txt[1].lower())
    )

    # Split reviews on any white space character
    id_words_rdd = id_txt_lower.map(
      lambda id_txt: (id_txt[0], id_txt[1].split())
    )


    def remove_stopwords(word_list: list[str]) -> list[str]:
      no_stopwords = []

      for word in word_list:
        if word not in stopwords.words('english'):
          no_stopwords.append(word)

      return no_stopwords

    # Remove stop words
    no_stopwords_rdd = id_words_rdd.map(
      lambda id_words: (id_words[0], remove_stopwords(id_words[1]))
    )
  \end{minted}
  %
  Further down the list, we notice stop words containing an apostrophe, such as
  \texttt{you're}, \texttt{you've}, \texttt{you'll}: this is why punctuation
  marks have not been removed right away. Now that the dataset no longer
  contains such stop words, punctuation can be removed, and the procedure to
  get rid of stop words is run again for good measure (in the previous
  execution, some stop words could have been missed because adjacent to
  punctuation). \\
  %
  \begin{minted}[
    frame=single,
    framesep=2mm,
    baselinestretch=1.2,
    bgcolor=LightGray,
    fontsize=\footnotesize,
    linenos
  ]{python}
    import string


    def remove_punctuation(word_list: list[str]) -> list[str]:
      no_punctuation = []

      for word in word_list:
        no_punctuation.append(word.translate(
          str.maketrans('', '', string.punctuation))
        )

      return no_punctuation

    # Remove punctuation
    no_punctuation = no_stopwords_rdd.map(
      lambda id_words: (id_words[0], remove_punctuation(id_words[1]))
    )
  \end{minted}
  %
  Then we remove numeric characters, and we make the search space more uniform
  by stemming words: once again, we leverage the “Natural Language Toolkit”
  package to extract the \textit{root} of each word. For instance, the root of
  word \textit{looking} is \textit{look}. \\
  %
  \begin{minted}[
    frame=single,
    framesep=2mm,
    baselinestretch=1.2,
    bgcolor=LightGray,
    fontsize=\footnotesize,
    linenos
  ]{python}
    import re


    def is_numeric(word: str) -> bool:
      numeric_pattern = re.compile(r'^[0-9]+$')
      return bool(numeric_pattern.match(word))

    # Remove words containing numbers
    no_numbers_rdd = no_stopwords.map(
      lambda id_words: (
        id_words[0],
        [word for word in id_words[1] if not is_numeric(word)]
      )
    )
  \end{minted}
  %
  \begin{minted}[
    frame=single,
    framesep=2mm,
    baselinestretch=1.2,
    bgcolor=LightGray,
    fontsize=\footnotesize,
    linenos
  ]{python}
    from nltk.stem import PorterStemmer

    nltk.download('punkt')
    stemmer = PorterStemmer()

    def to_stem(word_list: list[str]) -> list[str]:
      stems = []

      for word in word_list:
      stems.append(stemmer.stem(word))

      return stems


    # Extract stem from words
    stem_rdd = no_numbers_rdd.map(
      lambda id_words: (id_words[0], to_stem(id_words[1]))
    )
  \end{minted}
  %
  The last pre-processing step requires to select shingles. There are different
  examples of shingles, and the one used in this project is called
  \textit{k-gram}. K-grams are windows with a size of \( k \) characters, that
  slide over each review: for this reason we first need to join back together
  words that belong to the same review, putting a single space between each of
  them. \\
  %
  \begin{minted}[
    frame=single,
    framesep=2mm,
    baselinestretch=1.2,
    bgcolor=LightGray,
    fontsize=\footnotesize,
    linenos
  ]{python}
    def join_words(word_list: list[str]) -> list[str]:
      joined_words = ''

      for word in word_list:
        joined_words = joined_words + word
        if word != word_list[-1]:
          joined_words = joined_words + ' '

        return joined_words

    # Put words making up reviews back together
    cleaned_rdd = stem_rdd.map(
      lambda id_txt: (id_txt[0], join_words(id_txt[1]))
    )
  \end{minted}
  %
  The choice of \( k \) has a big impact on the algorithm that follows: we want
  some \( k \) large enough that the probability of two reviews to contain the
  same k-gram is negligible. In order to estimate this probability, we compute
  the average length \( l \) of the reviews: say \( l = 555 \). We should
  consider all the \( 26 \) letters of the English alphabet, but we take into
  account that some of them have a very small frequency by pretending we only
  have \( 20 \) different letters (or \( 19 \) letters, and a white space
  character). Now we compute the average amount of times \( t \) we expect to
  see a generic k-gram of length \( k \) over an alphabet of \( 20 \) letters:
  \[
    t = \frac{ l }{ { 20 }^k }
  \]
  Plugging in the value of \( l \) we got before,
  \[
    t = \frac{ 555 }{ { 20 }^k }
  \]
  Clearly, \( t \) decreases exponentially with \( k \): we could pick
  \( k = { 10 }^5 \), but then the algorithm would most likely detect only
  exact duplicates; instead, with some relatively small \( k \) such as
  \( k = 4 \), we get
  \begin{align*}
    t
    & = \frac{ 555 }{ { 20 }^4 } \\
    & \approx 0.004
  \end{align*}
  This means that on average we expect to see the same shingle in four elements
  when considering a sample of size \( 1000 \). \\
  %
  \begin{minted}[
    frame=single,
    framesep=2mm,
    baselinestretch=1.2,
    bgcolor=LightGray,
    fontsize=\footnotesize,
    linenos
  ]{python}
    def extract_kgrams(s: str, k: int) -> list[str]:
      kgrams = []

      for i in range(len(s) - k + 1):
        kgram = s[i : i + k]
        kgrams.append(kgram)

      return kgrams


    k = 5

    # Turn reviews into lists of k-grams
    kgrams_rdd = cleaned_rdd.map(
      lambda id_txt: (id_txt[0], extract_kgrams(id_txt[1], k))
    )
  \end{minted}
  %
  Finally, we can easily tell that not all possible combinations of
  \( { 20 }^k \) characters form a meaningful word, or part of word: thus we
  compress the representation of the k-grams by applying some hash function to
  each of them, and use the resulting hash values instead of the k-grams
  themselves. \\
  %
  \begin{minted}[
    frame=single,
    framesep=2mm,
    baselinestretch=1.2,
    bgcolor=LightGray,
    fontsize=\footnotesize,
    linenos
  ]{python}
    import hashlib
    from typing import Callable

    # Family of hash functions depending on parameter `i`
    def create_hash_function(i: int, result_size: int = 4):
      salt = str(i).encode('utf-8')

      def hash_string(input_string):
        data = str(input_string).encode('utf-8') + salt
        hashed = hashlib.sha256(data).hexdigest()

        return hashed[:result_size]

      return hash_string


    def hash_kgrams(
        kgrams: list[str], hash_func: Callable[[str], str]) -> list[str]:

      hashed_kgrams = []

      for kgram in kgrams:
        hashed_kgrams.append(hash_func(kgram))

      return hashed_kgrams


    hash_func = create_hash_function(123)

    # Hash shingles
    hashed_kgrams = kgrams_rdd.map(
        lambda kgram: (kgram[0], hash_kgrams(kgram[1], hash_func))
    )
  \end{minted}
  %

  \newpage
  \section{Jaccard similarity}
  Let \( \mathcal D_i \) and \( \mathcal D_j \) be two non-empty documents
  encoded as sets of shingles. The Jaccard similarity
  \( J \! : \mathcal X^2 \to [ 0, \, 1 ] \), with \( \mathcal X \) being the
  space where \( \mathcal D_i \) and \( \mathcal D_j \) live, is defined as
  \[
    J ( \mathcal D_i, \, \mathcal D_j ) =
    \frac{ \lvert \mathcal D_i \cap \mathcal D_j \rvert }{ \lvert \mathcal D_i \cup \mathcal D_j \rvert } \mathrm ,
  \]
  that is, the size of the intersection between \( \mathcal D_i \) and
  \( \mathcal D_j \) over the size of the union between \( \mathcal D_i \) and
  \( \mathcal D_j \). \\
  If \( \mathcal D_i \) and \( \mathcal D_j \) have \emph{all} words in common,
  i.e. \( \mathcal D_j = \mathcal D_i \), their Jaccard similarity is one:
  \begin{align*}
    \mathcal D_j = \mathcal D_i \implies J ( \mathcal D_i, \, \mathcal D_i )
    & = \frac{ \lvert \mathcal D_i \cap \mathcal D_i \rvert }{ \lvert \mathcal D_i \cup \mathcal D_i \rvert } \\
    & = \frac{ \lvert \mathcal D_i \rvert }{ \lvert \mathcal D_i \rvert } \\
    & = 1
  \end{align*}
  On the other hand, if \( \mathcal D_i \) and \( \mathcal D_j \) have
  \emph{no} words in common, i.e. \( \mathcal D_i \) and \( \mathcal D_j \) are
  disjoint, their Jaccard similarity is zero:
  \begin{align*}
    \mathcal D_i, \, \mathcal D_j \text{ disjoint } \implies
    J ( \mathcal D_i, \, \mathcal D_j )
    & = \frac{ \lvert \mathcal D_i \cap \mathcal D_j \rvert }{ \lvert \mathcal D_i \cup \mathcal D_j \rvert } \\
    & = \frac{ \lvert \emptyset \rvert }{ \lvert \mathcal D_i \cup \mathcal D_j \rvert } \\
    & = 0
  \end{align*}

  \section{Characteristic matrix}
  Given that k-grams are a set of sliding and overlapping windows, the amount
  of space we would need to store them is much larger than what is needed to
  store the original documents, even if the shingles are compressed. \\
  To sidestep this issue, we can think in terms of matrices: instead of storing
  k-grams, we build a boolean matrix the rows of which are indexed by document
  identifiers, and the columns by all possible k-grams; element
  \( ( i, \, j ) \) of this matrix is one if k-gram \( i \) appears in document
  \( j \), zero otherwise. Such a matrix is called “characteristic matrix.”
  %
  \begin{figure}[H]
    \[
      \begin{array}{l|rrrrr}
        & \mathcal D_1 & \mathcal D_2 & \mathcal D_3 & \mathcal D_4 & \dots \\
        \hline
        S_1 &   0 &   1 &   0 &   1 & \\
        S_2 &   1 &   0 &   0 &   1 & \\
        S_3 &   1 &   1 &   1 &   0 & \\
        S_4 &   0 &   0 &   1 &   1 & \\
        \vdots &     &     &     &     &
      \end{array}
    \]
    \caption*{An example of characteristic matrix.}
  \end{figure}
  %
  Now we could easily compute the Jaccard similarity of two documents by fixing
  the corresponding columns and counting the amount \( x \) of rows where two
  \( 1 \)'s appear: this is the size of the intersection between the documents;
  then, we count the amount \( x + y \) of rows where one or two \( 1 \)'s
  appear, which represents the size of the union between the documents, and
  divide the former by the latter. For instance, if we only consider the four
  shingles displayed in the example above,
  \[
    J ( \mathcal D_1, \, \mathcal D_2 ) = \frac{ 1 }{ 3 }
  \]
  since the \( S_3 \) row is the only one containing two \( 1 \)'s, while
  \( S_1, S_2, S_3 \) contains at least a \( 1 \). \\
  In practice, the characteristic matrix is usually very sparse, so we may
  first want to compress it, possibly without forgoing the ability to compute a
  reasonable approximation of the Jaccard similarity without having to undo the
  compression. This is where the concept of MinHash can help us.

  \section{MinHash}
  MinHash is a function mapping documents to shingles. \\
  The general schema for computing a MinHash function starts by applying some
  permutation to the matrix rows, that is to the shingles.
  %
  \begin{figure}[H]
    \[
      \begin{array}{l|rrrrr}
        & \mathcal D_1 & \mathcal D_2 & \mathcal D_3 & \mathcal D_4 & \dots \\
        \hline
        S_1 &   0 &   1 &   0 &   1 & \\
        S_2 &   1 &   0 &   0 &   1 & \\
        S_3 &   1 &   1 &   1 &   0 & \\
        S_4 &   0 &   0 &   1 &   1 & \\
        \vdots &     &     &     &     &
      \end{array}
      % \quad \underset{ \mathrm{permutation} }{ \longrightarrow } \quad
      \quad \underset{ \sigma_1 }{ \longrightarrow } \quad
      \begin{array}{l|rrrrr}
        & \mathcal D_1 & \mathcal D_2 & \mathcal D_3 & \mathcal D_4 & \dots \\
        \hline
        S_2 &   1 &   0 &   0 &   1 & \\
        S_1 &   0 &   1 &   0 &   1 & \\
        S_3 &   1 &   1 &   1 &   0 & \\
        S_4 &   0 &   0 &   1 &   1 & \\
        \vdots &     &     &     &     &
      \end{array}
    \]
    \caption*{
      The characteristic matrix, before and after applying permutation
      \( \sigma_1 \).
    }
  \end{figure}
  %
  Then, for each document we scan the corresponding column of the permuted
  matrix, and stop at the first \( 1 \) we encounter: the row we are in is
  indexed by some shingle, and this is the MinHash value of the document.
  %
  \begin{figure}[H]
    \[
      \begin{array}{l|lllll}
        & D_1 & D_2 & D_3 & D_4 & \dots \\
        \hline
        \sigma_1 & S_2 & S_1 & S_3 & S_1 & \\
      \end{array}
    \]
    \caption*{The result of applying a MinHash function once.}
  \end{figure}
  This process can be repeated a number of times, each time with a different
  permutation.
  %
  \begin{figure}[H]
    \[
      \begin{array}{l|lllll}
        & \mathcal D_1 & \mathcal D_2 & \mathcal D_3 & \mathcal D_4 & \dots \\
        \hline
        \sigma_1 & S_2 & S_1 & S_3 & S_1 & \\
        \sigma_2 & S_3 & S_1 & S_4 & S_1 &
      \end{array}
    \]
    \caption*{The result of applying MinHash twice.}
  \end{figure}
  %
  The resulting matrix is called “signature matrix”: it has the same amount of
  columns as the characteristic matrix, but a number of rows that is possibly
  smaller, as it depends on how many permutations we have used. \\
  Provided the permutations have been extracted uniformly at random among all
  available permutations of shingles, the following holds:
  \[
    \mathbb P ( H ( \mathcal D_i ) = H ( \mathcal D_j ) ) =
    J ( \mathcal D_i, \, \mathcal D_j ) \mathrm .
  \]
  That is, the Jaccard similarity between documents \( \mathcal D_i \) and
  \( \mathcal D_j \) can be estimated with the probability that a random
  MinHash function \( H \) computed on \( \mathcal D_i \) is the same as
  \( H \) on \( \mathcal D_j \). In order to prove this, let us consider the
  different type of rows we may encounter within the characteristic matrix.
  \begin{itemize}
    \item \( x \)-type rows: the row contains two \( 1 \)'s, so the
    corresponding shingle belongs to both \( \mathcal D_i \) and
    \( \mathcal D_j \)
    \item \( y \)-type rows: the row contains one \( 1 \), so the shingle
    belongs to either \( \mathcal D_i \) or \( \mathcal D_j \)
    \item \( z \)-type rows: the row contains no \( 1 \), so the shingle
    belongs to neither \( \mathcal D_i \) nor \( \mathcal D_j \)
  \end{itemize}
  Basically, fixed two documents and a row, we are considering all possible
  combinations in which \( 0 \)'s and \( 1 \)'s may appear while ignoring their
  relative order. \\
  Now, let \( x, \, y, \, z \) be the number of \( x \)-, \( y \)-,
  \( z \)-type rows respectively: we can compute the Jaccard similarity between
  \( \mathcal D_i \) and \( \mathcal D_j \) as
  \[
    J ( \mathcal D_i, \, \mathcal D_j ) = \frac{ x }{ x + y }
  \]
  To compute probability
  \( \mathbb P ( H ( \mathcal D_i ) = H ( \mathcal D_j ) ) \) instead, we
  simply consider the structure of any MinHash function: the columns of the
  characteristic matrix corresponding to \( \mathcal D_i \) and
  \( \mathcal D_j \) are scanned until a \( 1 \) is found. The event
  \( H ( \mathcal D_i ) = H ( \mathcal D_j ) \) holds whenever the same row
  contains \( 1 \) in both columns, otherwise the scanning would just continue
  for at least a column, and the value of \( H ( \mathcal D_i ) \) would be
  different than that of \( H ( \mathcal D_j ) \). So we are interested in the
  probability that the first non \( z \)-type row is of type \( x \): by
  assumption, the permutations are drawn uniformly at random, so we can compute
  the aforementioned probability via the relative frequency interpretation.
  \[
    \mathbb P ( H ( \mathcal D_i ) = H ( \mathcal D_j ) ) = \frac{ x }{ x + y }
  \]
  All in all, we have that
  \[
    \mathbb P ( H ( \mathcal D_i ) = H ( \mathcal D_j ) ) =
    J ( \mathcal D_i, \, \mathcal D_j )
  \]

  Although the space problem is more or less taken care of, we still have a
  time problem, as we need to compute the Jaccard similarity for all possible
  pairs within a massive dataset. \\

  Applying a MinHash function directly in big data contexts is a bad idea: not
  only would we need to produce a permutation as big as the amount \( n \) of
  rows in the characteristic matrix, but this would also need to be sorted
  according to the permutation, a process which cannot happen faster than
  \( n log ( n ) \). \\
  Instead, we leverage hash functions once again. Just like before, the hash
  function takes a shingle as input, and returns some bucket as output: in
  order to avoid collisions, we would need a perfect hash function; moreover,
  a family of hash functions is actually needed to avoid making the process too
  slow. \\
  At the beginning, we initialize the entire signature matrix with some big
  number, e.g. \( + \infty \):
  \[
    \forall \, i = 1, \, \dots, \, I, \quad \forall \, c = 1, \, \dots, \, C
    \qquad \mathrm{ sig } ( i, \, c ) = + \infty \mathrm ,
  \]
  where \( \mathrm{ sig } \) is the signature matrix of size \( I \times C \).
  \\
  Then, for each shingle in the characteristic matrix, we compute its hash
  value according to all members of the family: if the corresponding entry of
  the characteristic matrix is \( 1 \), we set the current element of the
  signature matrix to the minimum between the current value and the hash value.
  \\
  Given the number \( R \) of shingles,
  \( \forall \, r = 0, \, \dots, \, R - 1 \)
  \begin{itemize}
    \item compute \( h_i \, ( r ) \quad \forall \, i = 1, \, \dots, \, I \)
    \item \( \forall \, c = 1, \, \dots, \, C \)
    \begin{itemize}
      \item[] if \( \mathrm{ char } ( r, \, c ) = 1 \):
      \begin{itemize}
        \item[] \( \forall \, i = 1, \, \dots, \, I \quad
        \mathrm{ sig } ( i, \, c ) =
        \min ( \mathrm{ sig } ( i, \, c ), h_i \, ( r ) ) \)
      \end{itemize}
    \end{itemize}
  \end{itemize}
  Albeit with some differences, related to the fact we do not actually need a
  characteristic matrix, the following instructions can be used to apply a
  MinHash function to the list of k-grams, exploiting the same hash family we
  created earlier.
  %
  \begin{minted}[
    frame=single,
    framesep=2mm,
    baselinestretch=1.2,
    bgcolor=LightGray,
    fontsize=\footnotesize,
    linenos
  ]{python}
    def min_hash(kgrams: list, i: int) -> int:
      #   instantiate some hash function depending on `i` (which will be the
      # current row index)
      hash_func = create_hash_function(i)

      # initialize the current element of the signature matrix
      min_signature = float('inf')

      for kgram in kgrams:
        #   Turn the hash value, which can be interpreted as a hexadecimal
        # number, into an integer
        hash_signature = int(hash_func(kgram), 16)

        # Replace the current element if the hash value is smaller
        if hash_signature < min_signature:
          min_signature = hash_signature

      return min_signature


      n = 192  # Number of elements in the signature matrix

      signature_matrix = hashed_kgrams.map(
        lambda hashed_kgram:
        (hashed_kgram[0], [min_hash(hashed_kgram[1], i) for i in range(n)])
      )
  \end{minted}
  %

  \section[LSH]{Locality-sensitive hashing}
  An approximate algorithm to solve the time problem is called
  “locality-sensitive hashing”. \\
  Given the signature matrix with \( n \) elements, we divide it into \( b \)
  bands and \( r \) rows per band such that \( b \cdot r = n \).
  %
  \begin{minted}[
    frame=single,
    framesep=2mm,
    baselinestretch=1.2,
    bgcolor=LightGray,
    fontsize=\footnotesize,
    linenos
  ]{python}
    # Type annotation has been omitted for simplicity
    def divide_signature_matrix(doc_id_sig, b, r):
      doc_id = doc_id_sig[0]
      doc_sig = doc_id_sig[1]

      hash_func = create_hash_function(123)

      out = []

      for i in range(b):
        band_id = i
        offset = i * r

        column = ''.join(str(n) for n in doc_sig[offset : offset + r])
        # Hash function that takes vectors of `r` integers
        bucket_id = hash_func(column)

        #   Same hash function for all the bands, separate bucket array for
        # each band
        key = (band_id, bucket_id)
        out.append((key, doc_id))

      return out
  \end{minted}
  %
  Now we only select pairs of documents such that all their rows within at
  least one band are the same: if we cannot find any such band, the documents
  are most likely too different and their Jaccard similarity is very low; we do
  not need to process them further. \\
  %
  \begin{minted}[
    frame=single,
    framesep=2mm,
    baselinestretch=1.2,
    bgcolor=LightGray,
    fontsize=\footnotesize,
    linenos
  ]{python}
    #   Approximation: documents having at least the rows in one band in common
    # surely hash to the same bucket, but there could be collisions depending
    # on how close the hash function is to the perfect one
    # `hashed_bands` is the output of `divide_signature_matrix`
    candidates = hashed_bands.groupByKey().map(
      lambda x: (x[0], list(x[1]))
    )
  \end{minted}
  %
  The probability for a pair to be selected depends on both \( b \) and
  \( r \): in particular, it is \( 1 - ( 1 - s^r )^b \), \( s \) being the
  probability \( \mathcal D_i \) and \( \mathcal D_j \) agree in one row, that
  is their Jaccard similarity. To prove this, we note the probability of
  \( \mathcal D_i \) and \( \mathcal D_j \) to agree in an entire band, made up
  of \( r \) rows, is \( s^r \); the complementary event is that
  \( \mathcal D_i \) and \( \mathcal D_j \) do not agree in one band, and its
  probability is \( 1 - s^r \). The probability of \( \mathcal D_i \) and
  \( \mathcal D_j \) not to agree in \textit{any} of the \( b \) bands is
  \( ( 1 - s^r )^b \). Finally, the complementary of this event, i.e.
  \( \mathcal D_i \) and \( \mathcal D_j \) agree in at least one band, has
  probability \( p ( s ) := 1 - ( 1 - s^r )^b \). \\
  Since we want to determine whether a pair is selected or not, we normally
  apply some threshold to function \( p ( s ) \): this can be obtained by
  considering whatever \( s \) causes the steepest change in the function
  value, so we compute it by checking when the second derivative of
  \( p( s ) \) is zero. Provided the number \( r \) of rows is large enough,
  the \( s \) satisfying the property is
  \[
    \left ( \frac 1 b \right )^{ \slashfrac 1 r } \mathrm .
  \]
  The number \( n \) of rows in the signature matrix is fixed, so we have two
  variables: either we choose some pair \( ( b, \, r ) \) whose
  product is \( n \), or we choose \( s \) and compute the resulting
  \( ( b, \, r ) \). The second approach is usually chosen in practice: we fix
  some \( t \) so that only pairs of documents whose similarity is larger than
  \( t \) are selected, and obtain the desired \( ( b, \, r ) \) as a result.
  \\
  %
  \begin{minted}[
    frame=single,
    framesep=2mm,
    baselinestretch=1.2,
    bgcolor=LightGray,
    fontsize=\footnotesize,
    linenos
  ]{python}
    def func(x):
      b, r = x
      eq1 = (1 / b)**(1 / r) - t
      eq2 = b * r - n

      return [eq1, eq2]

    solutions = fsolve(func, [1, 1])
    b_solution, r_solution = solutions
  \end{minted}
  %
  Remember that LSH is a family of approximate algorithms: in other words, the
  result we obtain can be different that the actual result. When it comes to
  classification, which is the kind of problem we have to deal with after
  introducing a threshold \( t \) on the similarity, this translates to the
  presence of false positives and false negatives: the former are a set of
  similar documents according to the approximate approach, but dissimilar
  according to the exact approach; the latter are instead claimed to be
  dissimilar by LSH, and are actually similar when following the exact
  approach.

  \newpage

  \section{Scaling}
  All operations discussed here, except of course for the initial loading of
  the dataset, leverage the Apache Spark framework, and in particular its
  implementation of \textbf resilient \textbf distributed \textbf datasets
  (RDDs) and MapReduce. \\
  Roughly speaking, the RDD is distributed across different nodes in a cluster
  of machines, and each of them only works on its portion of the dataset. Only
  operations compliant with the MapReduce framework can be executed on RDDs:
  \texttt{map} transforms data, while \texttt{reduce} aggregates it. \\
  The main focus of Spark is scalability, so if used correctly it should not
  fall short. Unfortunately, exploiting clusters of machines means we have to
  rely on some network, which becomes the main bottleneck: for this reason, the
  so-called “shuffling” should be avoided as much as possible. Shuffling occurs
  whenever the cluster nodes interact with each other across the network; the
  most trivial example of shuffling occurs when we ask the RDD for its
  contents: this requires all nodes to send their dataset portion to one
  machine (the “driver”) before it can be displayed as a whole. \\
  The free tier of Google Colab can never show such an issue, because the
  cluster it provides us with is made up by a single node. This clearly
  cripples the performance of Spark, so we have no other choice but to throw
  away the majority of the dataset, retaining only a small fraction over which
  the operations are performed. A first possibility of scaling is thus given by
  the size of the dataset, as it could easily grow given a larger cluster. \\
  Another interesting point is the following operation, by far the slowest one:
  %
  \begin{minted}[
    frame=single,
    framesep=2mm,
    baselinestretch=1.2,
    bgcolor=LightGray,
    fontsize=\footnotesize,
    linenos
  ]{python}
    candidates = hashed_bands.groupByKey().map(
      lambda x: (x[0], list(x[1]))
    )
  \end{minted}
  %
  We have already ruled out the possibility that this operation is so slow
  because of shuffling, and we are left with \texttt{groupByKey} and
  \texttt{map}: the first aggregates pairs \texttt x with the same key
  \texttt{x[0]}, putting the corresponding \texttt{x[1]} into some
  \texttt{Iterable}; the second then casts this \texttt{Iterable} into a
  \texttt{list}. \texttt{groupByKey} is most likely the culprit: in order to
  compute it, Spark first sorts the dataset by key; under normal conditions,
  and with such a small dataset, this should not be cause of concern, but the
  key is not a simple object, being a pair of integer and string. It could be
  interesting to see how this operation translates in an actually distributed
  scenario, where both sorting and shuffling happen.

  \section{Experiments}
  The remaining part of the experiment is pretty straightforward. Once we have
  the lists of documents with all rows within at least one band in common, we
  filter out any list with less than two elements, as they contain documents
  whose similarity with the others is smaller than threshold \( t \).
  %
  \begin{minted}[
    frame=single,
    framesep=2mm,
    baselinestretch=1.2,
    bgcolor=LightGray,
    fontsize=\footnotesize,
    linenos
  ]{python}
    filtered_candidates = candidates.filter(
      lambda collision_list: len(collision_list[1]) > 1
    )
  \end{minted}
  %
  Starting from these lists, we extract pairs of candidate documents:
  %
  \begin{minted}[
    frame=single,
    framesep=2mm,
    baselinestretch=1.2,
    bgcolor=LightGray,
    fontsize=\footnotesize,
    linenos
  ]{python}
    import itertools


    def get_pairs(collisions_list: list) -> tuple:
      pair_list = []

      for pair in itertools.combinations(collisions_list, 2):
        pair_list.append(pair)

      return tuple(pair_list)


    candidate_docs = filtered_candidates.flatMap(
      lambda pairs: get_pairs(pairs[1])
    ).distinct()
  \end{minted}
  %
  Given these pairs of identifiers, we can easily retrieve the corresponding
  documents from the original RDD. \\
  In principle, the experiment could end here: if we were working on an
  actually massive dataset, it would not be possible to compare the result of
  the approximate algorithm with the one of the exact algorithm. Since we can,
  we compute the Jaccard similarity between pairs of documents within the
  signature matrix.
  %
  \begin{minted}[
    frame=single,
    framesep=2mm,
    baselinestretch=1.2,
    bgcolor=LightGray,
    fontsize=\footnotesize,
    linenos
  ]{python}
    # `candidate_list` contains the identifier of all similar documents
    filtered_signature_matrix = signature_matrix.filter(
      lambda id_sig: id_sig[0] in candidate_list
    )

    # `filter` is needed to exclude pairs containing the same identifier
    cartesian_product = (
      filtered_signature_matrix.cartesian(filtered_signature_matrix)
                               .filter(lambda id_minhash:
                                       id_minhash[0][0] != id_minhash[1][0])
    )

    def jaccard_sim(set1: set, set2: set) -> float:
      intersection_length = len(set1.intersection(set2))
      union_length = len(set1.union(set2))

      return float(intersection_length) / float(union_length)


    #   Each triple within the following RDD contains two document identifiers
    # and their Jaccard similarity
    jaccard_rdd = cartesian_product.map(
        lambda p: (p[0][0], p[1][0], jaccard_sim(set(p[0][1]), set(p[1][1])))
    )
  \end{minted}
  %
  Finally, we remove those pairs of documents whose Jaccard similarity is
  smaller than threshold \( t \), and we compute the intersection between the
  remaining pairs and the output of the approximate algorithm.
  %
  \begin{minted}[
    frame=single,
    framesep=2mm,
    baselinestretch=1.2,
    bgcolor=LightGray,
    fontsize=\footnotesize,
    linenos
  ]{python}
    similar_docs = jaccard_rdd.filter(lambda p: p[2] >= t)
    lsh_jaccard_intersection = similar_docs.intersection(candidate_docs)
  \end{minted}
  %
  On this particular dataset, the comparison between exact and approximate
  algorithms is only possible with thresholds smaller than \( 0.6 \), as no
  pair of reviews with a similarity of \( 0.7 \) exists (and there are only two
  pairs with a similarity of \( 0.6 \)). \\
  Even so, despite some attempts with different values of hyperparameters, the
  intersection between results is empty when considering pairs of identifiers.

  \section{Results}
  The main takeaway of the experiments is that the dataset contains no
  duplicates, at least in the small subset that was considered. \\
  Fixing the length of \( k \)-grams to \( 5 \), the number of signatures per
  review to \( 256 \) and the similarity threshold to \( 0.6 \), the most
  similar pair of documents is the following. \\

  \textbf{Review} \texttt{o5O3CYV79dTN8O9mFijmFA}: \\
  «\textit{Finally went to Delicia! My boyfriend and I went there for date
  night and had a fantastic time! At 7:15 on a Tuesday, the place was lively
  but there was no wait for a table for 2. The decor is cool and romantic, the
  host was very gracious, and our waiter, Shawn, was friendly and made good
  recommendations. I loved my margarita, and my boyfriend's sangria was tasty -
  nice and winey! I also tried the mojito. It was good... but tasted like
  Double Mint gum to me - not in a bad way but not in a way I liked. If you
  tasted the same thing, please, PLEASE message me so I feel less crazy!} \\

  \textit{Anyway, the queso fundido appetizer was delicious and unique (the
  chips were great too). For dinner, he got the beef empanadas and I got the
  duck enchiladas. When we go back to Delicia again (and we will), I want to
  try the scallops, but the duck enchiladas, rice and beans were so good it's
  hard to imagine ordering anything else! My boyfriend really enjoyed the flaky
  empanadas, as well. We chose wisely.} \\

  \textit{Finally, we tried the Fire and Ice cocktail, about which I've heard
  plenty of raves. Well, it wasn't really our thing. I mean, it's a quality
  cocktail, and the chile ice ball is very cool, but it tastes like drinking
  jalapenos! I guess some people like that, so power to them. But while I
  didn't love the cocktail, I'm glad it's something you can get in Indy, you
  know? Delicia is a great restaurant that helps take Indy's food and cocktail
  scene up another notch.»} \\

  \textbf{Review} \texttt{SbeoXuUHy1xgL-o7Urpkeg}: \\
  «\textit{`Top donuts in the country?'} \\

  \textit{Okay. Whenever I read something that says 'top ten blah blah of
  something in the us or state of' my wife and I are extremely hesitant because
  it sounds like a gimmick. After coming to experience this place first hand I
  can understand why you would make that statement. We visited on a Thursday
  10:30amish with a line of 5 people. They don't serve sliders until 11am and
  this gave us a great opportunity to do both separately.} \\

  \textit{Ambience:} \\
  \textit{Alot of character and with a ton of regulars. They were on a first
  name basis with half the people in the place. Located on magazine st in the
  garden district. Artsy stuff. Nice street to walk, explore, and shop.} \\

  \textit{Food:} \\
  \textit{They categorize their donuts into 3 categories to simplify pricing.
  Reg/fancy/more fancy. This is brilliant as it speeds up the ordering but they
  still carry variety on their menu.} \\

  \textit{Glazed donut 4/5} \\
  \textit{This was a good, well balanced donut. Not overly sweet but had
  characteristics of what you expect from a donut. I prefer my donuts a bit
  under cooked but this didn't change the pillow like texture one would desire.
  The glaze was delicate yet slept on the donut. It didn't over power it at
  all.} \\

  \textit{French Toast donut 5/5} \\
  \textit{My wife had this one and was very surprised by the balance this donut
  had. Generally, donuts are rich and over the top in these types of
  establishments, but the donuts made here tasted like they had a plan and they
  were going to make it their way and not please us the customer. I have alot
  of respect for these places.} \\

  \textit{Fried chicken slider 4/5} \\
  \textit{Small in stature but tasty, balanced and filling. Don't let this
  slider fool you. It makes a statement with all the pieces it carries: the
  slaw was amazing, the chicken fried and seasoned properly, to the pickled
  jalapeño. Great combination.} \\

  \textit{Pork belly slider 3/5} \\
  \textit{This slider is very popular among their clientele. Good slider.
  Picked onion brings nice tart and cuts richness of the belly, the herbs
  leaves brings a certain bitterness that contrasted the fattiness of the
  belly. The belly itself was properly cooked. Slow roasted ahead of time then
  heated up on the flat top for crust. Seasoned well and you get that
  concentration of the cure. I would have had less of a dry brine imo, but
  that's just me. My wife says I'm being picky.} \\

  \textit{Drink:} \\
  \textit{Cold brew. 4/5} \\
  \textit{I enjoyed this and honestly had to settle for this. They normally
  have the Nitro cold brew but it wasn't working today. On a Louisiana humid
  sunny day there is nothing like proper iced coffee.} \\

  \textit{Service:} \\
  \textit{Friendly but for sure they cater to the locals. They aren't here for
  tourists and they shouldn't baby the tourists. 5/5»} \\

  We can easily tell these reviews are not so similar: the general opinion is
  favorable and both talk about food (not all do), but one is concerned on the
  menu of some restaurant, while the other focuses on donuts. Given how the
  algorithm works, it is pretty normal that these reviews are different when in
  human-readable form: we could have two reviews with \( 60\% \) of words in
  common talking about completely different topics, but MinHash and locality
  sensitive hashing would not pick up on that; instead, they would only
  consider what makes them similar, and classify them as such if the threshold
  is permissive enough. In any case, the similarity between the reviews above,
  computed on the associated sets of signatures, is only \( 0.44 \) (despite
  the threshold of \( 0.6 \)), meaning the algorithm too sees them are more
  dissimilar than similar.

  \subsection*{Possible extensions}
  \textbf{Building a proper pipeline to easily run experiments}. The pipeline
  could also feature a step of model selection, looking for hyperparameters
  values that make the final result closer to the exact algorithm; of course,
  this would only be possible for dataset whose size allows to compute Jaccard
  similarity in a reasonable amount of time. Another option is to ignore
  similarity levels, and search for those values of hyperparameters that lead
  to the smallest difference between the set of similar pairs resulting from
  the approximated algorithm, and the set of similar pairs resulting from the
  exact algorithm. \\
  \textbf{Trying out different ways to embed documents}. We have seen TF-IDF
  can be leveraged instead of \( k \)-grams for making text documents easier to
  work with. For instance, we could encode reviews as the top \( 10 \) words
  ranked according to TF-IDF, or as the set of words whose TF-IDF exceeds a
  fixed threshold; in both cases, an additional hyperparameter is required, so
  the model selection step would require exponentially more time. Then it could
  be interesting to compare the results yielded by these different approaches.

\end{document}
